%---------- Inleiding ---------------------------------------------------------

% TODO: Is dit voorstel gebaseerd op een paper van Research Methods die je
% vorig jaar hebt ingediend? Heb je daarbij eventueel samengewerkt met een
% andere student?
% Zo ja, haal dan de tekst hieronder uit commentaar en pas aan.

%\paragraph{Opmerking}

% Dit voorstel is gebaseerd op het onderzoeksvoorstel dat werd geschreven in het
% kader van het vak Research Methods dat ik (vorig/dit) academiejaar heb
% uitgewerkt (met medesturent VOORNAAM NAAM als mede-auteur).
% 

\section{Inleiding}%
\label{sec:inleiding}

Om de kwaliteit van leven te verbeteren, heeft de VN zich ten doel gesteld om de voedselverspilling per hoofd van de bevolking tegen 2030 te halveren. Een correcte boekhouding en distributie van gekochte goederen vormt de basis van rationele consumptie.
Ondanks het bestaan ​​van een groot aantal voorraadbeheerprogramma's, blijven kleine bedrijven hun administratie bijhouden in Excel en op papier. Deze keuze is gebaseerd op de veelzijdigheid en het gebrek aan maandelijkse financiële kosten voor software. De nadelen van deze aanpak zijn de foutgevoeligheid, beperkte schaalbaarheid en beperkte integratie.
Dit proefschrift richt zich op de selectie van voorraadbeheersoftware die het beste aansluit bij de eisen van de EasyCoffee-koffiespeciaalzaakketen. Deze sector staat voor uitdagingen op het gebied van voorraadbeheer, traceerbaarheid van voorraden en snelle afschrijving van producten met een beperkte houdbaarheid.
Deze specifieke uitdagingen leiden tot diverse verliezen. Ze veroorzaken met name een stijging van de bedrijfskosten, een daling van de winst, overmatige inkoop, meer voedselverspilling en problemen met voorraadbeheer.
Het in dit project ontwikkelde prototype beoogt al deze problemen op te lossen. 

Deze taak is met name relevant in het midden- en kleinbedrijf, zoals zelfstandige restaurants en boetiekhotels, waar grootschalige ERP-oplossingen niet beschikbaar zijn.

\textbf{De doelgroep van dit onderzoek} zijn IT-professionals die betrokken zijn bij de digitalisering van bedrijfsprocessen in de horeca, evenals bedrijfsleiders die interne processen voor de opslag en boekhouding van producten en diensten willen optimaliseren.

\textbf{Probleem:} Kleine horecabedrijven missen vaak een gestructureerde aanpak voor voorraadbeheer. Dit leidt tot productverlies, inefficiënte inkoop en het onvermogen om op elk moment snel een inventarisatie uit te voeren. Excel- of papieren oplossingen die in dergelijke gevallen worden gebruikt, zijn niet schaalbaar en foutgevoelig.

\textbf{Centrale onderzoeksvraag:} Welke bestaande software is het meest geschikt voor een kleine horecazaak, bijvoorbeeld de IzyCoffee-koffieketen? En welke praktische verbeteringen kunnen worden aangebracht om ervoor te zorgen dat de software voldoet aan de eisen van de betreffende onderneming?

\textbf{Deel vragen} zijn:
\begin{itemize}
  \item Welke oplossingen zijn er al op de markt? 
  \item Welke software is het meest geschikt voor de IzyCoffee-koffieketen?
  \item Welke functies en features kunnen worden toegevoegd om de software aan een breder scala aan eisen te laten voldoen?
  \item Wat zijn de voordelen van de in dit proefschrift voorgestelde oplossing ten opzichte van de huidige boekhoudmethoden? 
  \item Welke verdere verbeteringen aan het boekhoudsysteem zijn mogelijk, rekening houdend met de inzet van informatietechnologie?
\end{itemize}

\textbf{Het doel van de studie} is het kiezen van de meest geschikte software voor efficiënte boekhouding van ontvangsten, balansen en voorraden in een magazijn biedt, evenals rolgebaseerde toegang tot functionaliteit voor verschillende categorieën gebruikers (managers, magazijnmedewerkers, koks) een ontwikkeling van een prototype (proof-of-concept) van verbeteringen software op basis van bestande code. Als bijkomend doel wordt de mogelijkheid tot uitbreiding van het systeem overwogen, rekening houdend met toekomstige integraties (bijvoorbeeld met POS- (point of sale) of analysemodules).

Het project is daarom van toegepaste aard en lost een specifiek probleem op met betrekking tot de boekhouding van materialen, met horecabedrijven als voorbeeld, met een focus op praktijkscenario's en typische beperkingen van kleine bedrijven.
%---------- Stand van zaken ---------------------------------------------------

\section{Literatuurstudie}%
\label{sec:literatuurstudie}

In 2015 stelden de Verenigde Naties Duurzame Ontwikkelingsdoelstelling 12.3 vast, die ernaar streeft de voedselverspilling per hoofd van de bevolking tegen 2030 te halveren ~\autocite{VN2024}. Deze beslissing onderstreept het belang van het aanpakken van voedselverspilling om voedselzekerheid te waarborgen, hulpbronnen duurzaam te gebruiken en de negatieve impact op het milieu te verminderen.

In het kader van deze doelstelling zijn verschillende sectoren, waaronder de horeca, begonnen met het implementeren van maatregelen die gericht zijn op het verminderen van voedselverspilling. Een studie gepubliceerd in het Journal of Cleaner Production \textcite{Cardenas2024} analyseert circulaire praktijken in de horeca, zoals het recyclen van restjes en menu-optimalisatie, die bijdragen aan het verminderen van afval en het bevorderen van duurzame consumptie.

Daarnaast heeft de Wereldtoerismeorganisatie een wereldwijde routekaart ontwikkeld voor het verminderen van voedselverspilling in de toeristische sector, met aanbevelingen voor het implementeren van duurzame praktijken in hotels en restaurants, waaronder personeelstraining en het gebruik van digitale tools om afval te monitoren \parencite{2023}.

Digitale transformatie speelt een sleutelrol in deze inspanningen. Een bibliometrisch onderzoek gepubliceerd in het International Journal of Hospitality Management identificeerde vier belangrijke onderzoeksgebieden op het gebied van digitale transformatie in de horeca: digitale adoptie, impact op stakeholders, determinanten van online aankopen en analyse van online reviews ~\textcite{Peng2024}. Deze gebieden benadrukken het belang van digitale oplossingen, zoals geavanceerde voorraadbeheersystemen, voor het verbeteren van de efficiëntie en duurzaamheid van horecabedrijven.

Het bereiken van de VN-doelstelling om voedselverspilling te verminderen, vereist daarom een ​​alomvattende aanpak die de implementatie van circulaire praktijken en digitale technologieën in de horeca omvat. Dit draagt ​​niet alleen bij aan de bescherming van het milieu, maar verbetert ook de operationele efficiëntie en de duurzaamheid van bedrijven.
% Voor literatuurverwijzingen zijn er twee belangrijke commando's:
% \autocite{KEY} => (Auteur, jaartal) Gebruik dit als de naam van de auteur
%   geen onderdeel is van de zin.
% \textcite{KEY} => Auteur (jaartal)  Gebruik dit als de auteursnaam wel een
%   functie heeft in de zin (bv. ``Uit onderzoek door Doll & Hill (1954) bleek
%   ...'')

%Je mag deze sectie nog verder onderverdelen in subsecties als dit de structuur van de tekst kan verduidelijken.

%---------- Methodologie ------------------------------------------------------
\section{Methodologie}%
\label{sec:methodologie}

Het doel van de vergelijkende analyse is om de meest geschikte bestaande software voor inventarisatie en logistiek van een keten van koffiebars te bepalen.

Hiertoe is het werk verdeeld in de volgende fasen:
\begin{itemize}
\item literatuuronderzoek om vergelijkbare gevallen te vinden, problemen waarmee kleine en middelgrote bedrijven te maken krijgen tijdens inventarisatie in de horeca en mogelijke oplossingen voor deze problemen (1-4 november)
\item het samenstellen van een lange lijst met bestaande software voor productinventarisatie (5-6 november)
\item het definiëren van vereisten voor een inventarisatiesysteem aan de hand van het voorbeeld van de IzyCoffee keten van koffiebars (1-6 november)
\item het rangschikken van vereisten met behulp van de MoSCoW-methode om de prioriteit van vereisten te bepalen, op basis waarvan de meest geschikte software wordt geselecteerd (6 november)
\item het selecteren van twee-drie softwareprogramma's die het beste aan de vereisten voldoen en deze met elkaar vergelijken (7 november)
\item het bepalen van de meest geschikte software voor zakelijke doeleinden (7 november)
\item het aantonen van de verschillen tussen de momenteel gebruikte methoden en de in het werk voorgestelde oplossing (8 november)
\item het identificeren van ontbrekende vereisten en suggesties voor het oplossen hiervan (9 november)
\item het implementeren van PoC in de geselecteerde software om deze te optimaliseren en te voldoen aan de noodzakelijke vereisten (10-15 november)
\item het formuleren van conclusies (16 november)
\item suggesties voor verdere verbetering en uitbreiding van de software (17 november)
\end{itemize}
%---------- Verwachte resultaten ----------------------------------------------
\section{Verwacht resultaat, conclusie}%
\label{sec:verwachte_resultaten}

Het resultaat van het onderzoek is dat moet worden bepaald welke van de bestaande programma's het beste voldoet aan de eisen van de keten van speciaalzaken in koffie IzyCoffee. Dit moet worden bepaald aan de hand van zowel opensourceprogramma's als betaalde versies.
Zo krijgt de potentiële gebruiker inzicht in hoe het programma voldoet en hoe de functionaliteit kan worden verbeterd, afhankelijk van de specifieke doelen en doelstellingen en de missie van het bedrijf. Het proefschrift omvat ook het aantonen van de voordelen van het gebruik van een doelprogramma ten opzichte van de methoden die worden gebruikt voor het vergelijken van echte data.
\textbf{Doelgroep van het project:} managementpersoneel en medewerkers van de IzyCoffee-keten van koffieshops en vergelijkbare bedrijven. De voorgestelde oplossing is waardevol als een lichtgewicht boekhoudoplossing zonder de noodzaak om complexe ERP te implementeren.
Het onderzoek biedt praktische waarde en vormt de basis voor een voorraadsysteem dat mogelijk in een specifieke instelling kan worden geïmplementeerd.
