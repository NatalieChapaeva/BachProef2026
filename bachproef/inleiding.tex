%%=============================================================================
%% Inleiding
%%=============================================================================

\chapter{\IfLanguageName{dutch}{Inleiding}{Introduction}}%
\label{ch:inleiding}

Om de kwaliteit van leven te verbeteren, heeft de VN zich ten doel gesteld om de voedselverspilling per hoofd van de bevolking tegen 2030 te halveren. Een correcte boekhouding en distributie van gekochte goederen vormt de basis van rationele consumptie.
Ondanks het bestaan ​​van een groot aantal voorraadbeheerprogramma's, blijven kleine bedrijven hun administratie bijhouden in Excel en op papier. Deze keuze is gebaseerd op de veelzijdigheid en het gebrek aan maandelijkse financiële kosten voor software. De nadelen van deze aanpak zijn de foutgevoeligheid, beperkte schaalbaarheid en beperkte integratie.
Dit proefschrift richt zich op de selectie van voorraadbeheersoftware die het beste aansluit bij de eisen van de EasyCoffee-koffiespeciaalzaakketen. Deze sector staat voor uitdagingen op het gebied van voorraadbeheer, traceerbaarheid van voorraden en snelle afschrijving van producten met een beperkte houdbaarheid.
Deze specifieke uitdagingen leiden tot diverse verliezen. Ze veroorzaken met name een stijging van de bedrijfskosten, een daling van de winst, overmatige inkoop, meer voedselverspilling en problemen met voorraadbeheer.
Het in dit project ontwikkelde prototype beoogt al deze problemen op te lossen. 

Deze taak is met name relevant in het midden- en kleinbedrijf, zoals zelfstandige restaurants en boetiekhotels, waar grootschalige ERP-oplossingen niet beschikbaar zijn.


\section{\IfLanguageName{dutch}{Probleemstelling}{Problem Statement}}%
\label{sec:probleemstelling}

Kleine horecabedrijven missen vaak een gestructureerde aanpak voor voorraadbeheer. Dit leidt tot productverlies, inefficiënte inkoop en het onvermogen om op elk moment snel een inventarisatie uit te voeren. Excel- of papieren oplossingen die in dergelijke gevallen worden gebruikt, zijn niet schaalbaar en foutgevoelig.

\textbf{De doelgroep van dit onderzoek} medewerkers van IzyCoffee: barista's en management. Dit bachelor proefschrift wekt ook de interesse van IT-professionals die betrokken zijn bij de digitalisering van bedrijfsprocessen in de horeca, evenals bedrijfsleiders die interne processen voor de opslag en boekhouding van producten en diensten willen optimaliseren.


\section{\IfLanguageName{dutch}{Onderzoeksvraag}{Research question}}%
\label{sec:onderzoeksvraag}

Welke bestaande software is het meest geschikt voor een kleine horecazaak, bijvoorbeeld de IzyCoffee-koffieketen? En welke praktische verbeteringen kunnen worden aangebracht om ervoor te zorgen dat de software voldoet aan de eisen van de betreffende onderneming?

\textbf{Deel vragen} zijn:
\begin{itemize}
    \item Welke oplossingen zijn er al op de markt? 
    \item Welke software is het meest geschikt voor de IzyCoffee-koffieketen?
    \item Welke functies en features kunnen worden toegevoegd om de software aan een breder scala aan eisen te laten voldoen?
    \item Wat zijn de voordelen van de in dit proefschrift voorgestelde oplossing ten opzichte van de huidige boekhoudmethoden? 
    \item Welke verdere verbeteringen aan het boekhoudsysteem zijn mogelijk, rekening houdend met de inzet van informatietechnologie?
\end{itemize}


\section{\IfLanguageName{dutch}{Onderzoeksdoelstelling}{Research objective}}%
\label{sec:onderzoeksdoelstelling}

Het doel van de studie is het kiezen van de meest geschikte software voor efficiënte boekhouding van ontvangsten, balansen en voorraden in een magazijn biedt, evenals rolgebaseerde toegang tot functionaliteit voor verschillende categorieën gebruikers (managers, magazijnmedewerkers, koks) een ontwikkeling van een prototype (proof-of-concept) van verbeteringen software op basis van bestande code. Als bijkomend doel wordt de mogelijkheid tot uitbreiding van het systeem overwogen, rekening houdend met toekomstige integraties (bijvoorbeeld met POS- (point of sale) of analysemodules).

Het project is daarom van toegepaste aard en lost een specifiek probleem op met betrekking tot de boekhouding van materialen, met horecabedrijven als voorbeeld, met een focus op praktijkscenario's en typische beperkingen van kleine bedrijven.


\section{\IfLanguageName{dutch}{Opzet van deze bachelorproef}{Structure of this bachelor thesis}}%
\label{sec:opzet-bachelorproef}

% Het is gebruikelijk aan het einde van de inleiding een overzicht te
% geven van de opbouw van de rest van de tekst. Deze sectie bevat al een aanzet
% die je kan aanvullen/aanpassen in functie van je eigen tekst.

De rest van deze bachelorproef is als volgt opgebouwd:

In Hoofdstuk~\ref{ch:stand-van-zaken} wordt een overzicht gegeven van de stand van zaken binnen het onderzoeksdomein, op basis van een literatuurstudie.

In Hoofdstuk~\ref{ch:methodologie} wordt de methodologie toegelicht en worden de gebruikte onderzoekstechnieken besproken om een antwoord te kunnen formuleren op de onderzoeksvragen.

In Hoofdstuk~\ref{ch:softwareselectie} werden de eisen van IzyCoffee voor een voorraadsysteem geformuleerd. Op basis hiervan werd de meest geschikte software geselecteerd.

In Hoofdstuk~\ref{ch:proof-of-concept} werden ontbrekende functies in de geselecteerde software geïdentificeerd. Aanpassingen aan de code van het geselecteerde programma werden ontwikkeld en geïmplementeerd om de functionaliteit uit te breiden en zo de bedrijfsdoelen beter te bereiken.

In Hoofdstuk~\ref{ch:conclusie}, tenslotte, wordt de conclusie gegeven en een antwoord geformuleerd op de onderzoeksvragen. Daarbij wordt ook een aanzet gegeven voor toekomstig onderzoek binnen dit domein.