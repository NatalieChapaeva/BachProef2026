%%=============================================================================
%% Inleiding
%%=============================================================================

\chapter{\IfLanguageName{dutch}{Inleiding}{Introduction}}%
\label{ch:inleiding}

Om de levenskwaliteit te verbeteren wil de Verenigde Naties de voedselverspilling per persoon tegen 2030 halveren. Een correcte opvolging en verdeling van aangekochte goederen vormt hiervoor een belangrijke basis.

Hoewel er vandaag veel voorraadbeheersystemen beschikbaar zijn, gebruiken kleine ondernemingen vaak nog Excel-bestanden en papieren documenten voor hun administratie. Deze keuze wordt meestal gemaakt omdat deze middelen flexibel zijn en geen vaste softwarekosten met zich meebrengen. Tegelijkertijd brengt deze aanpak nadelen met zich mee, zoals een grotere kans op fouten, beperkte uitbreidingsmogelijkheden en een gebrek aan koppelingen met andere systemen.

Deze bachelorproef focust op de selectie van voorraadbeheersoftware die aansluit bij de noden van de koffiespeciaalzaakketen IzyCoffee. Binnen deze sector zijn er specifieke uitdagingen, waaronder een moeilijk beheersbare voorraad, beperkte traceerbaarheid en producten met een korte houdbaarheid.

Deze uitdagingen zorgen voor verschillende problemen, zoals hogere kosten, een lagere winst, overmatige aankopen, meer voedselverspilling en een inefficiënt voorraadbeheer.

Het ontwikkelde prototype in dit project tracht deze problemen te verminderen door een praktische en gebruiksvriendelijke oplossing voor voorraadbeheer aan te bieden.

Dit thema is vooral relevant voor kleine en middelgrote ondernemingen, zoals restaurants en boetiekhotels, waar uitgebreide ERP-oplossingen vaak niet haalbaar zijn.


\section{\IfLanguageName{dutch}{Probleemstelling}{Problem Statement}}%
\label{sec:probleemstelling}

Kleine horecabedrijven missen vaak een gestructureerde aanpak voor voorraadbeheer. Dit leidt tot productverlies, inefficiënte inkoop en het onvermogen om op elk moment snel een inventarisatie uit te voeren. Excel- of papieren oplossingen die in dergelijke gevallen worden gebruikt, zijn niet schaalbaar en foutgevoelig.

\textbf{De doelgroep van dit onderzoek} medewerkers van IzyCoffee: barista's en management. Dit bachelor proefschrift wekt ook de interesse van IT-professionals die betrokken zijn bij de digitalisering van bedrijfsprocessen in de horeca, evenals bedrijfsleiders die interne processen voor de opslag en boekhouding van producten en diensten willen optimaliseren.


\section{\IfLanguageName{dutch}{Onderzoeksvraag}{Research question}}%
\label{sec:onderzoeksvraag}

Welke bestaande software is het meest geschikt voor een kleine horecazaak, bijvoorbeeld de IzyCoffee-koffieketen? En welke praktische verbeteringen kunnen worden aangebracht om ervoor te zorgen dat de software voldoet aan de eisen van de betreffende onderneming?

\textbf{Deel vragen} zijn:
\begin{itemize}
    \item Welke oplossingen zijn er al op de markt? 
    \item Welke software is het meest geschikt voor de IzyCoffee-koffieketen?
    \item Welke functies en features kunnen worden toegevoegd om de software aan een breder scala aan eisen te laten voldoen?
    \item Wat zijn de voordelen van de in dit proefschrift voorgestelde oplossing ten opzichte van de huidige boekhoudmethoden? 
    \item Welke verdere verbeteringen aan het boekhoudsysteem zijn mogelijk, rekening houdend met de inzet van informatietechnologie?
\end{itemize}


\section{\IfLanguageName{dutch}{Onderzoeksdoelstelling}{Research objective}}%
\label{sec:onderzoeksdoelstelling}

Het doel van deze studie is het selecteren van de geschiktste software voor een efficiënte administratie van ontvangsten, saldi en voorraad in een magazijn. Deze software moet tevens de toegang tot functionaliteiten beheren op basis van rollen voor verschillende gebruikerscategorieën (managers, magazijnmedewerkers, koks). Deze studie beoogt ook een prototype (proof of concept) te ontwikkelen voor softwareverbeteringen op basis van bestaande code. Als bijkomend doel wordt de mogelijkheid tot uitbreiding van het systeem overwogen, rekening houdend met toekomstige integraties (bijvoorbeeld met POS- (point of sale) of analysemodules).

Het project is daarom van toegepaste aard en lost een specifiek probleem op met betrekking tot de boekhouding van materialen, met horecabedrijven als voorbeeld, met een focus op praktijkscenario's en typische beperkingen van kleine bedrijven.


\section{\IfLanguageName{dutch}{Opzet van deze bachelorproef}{Structure of this bachelor thesis}}%
\label{sec:opzet-bachelorproef}

% Het is gebruikelijk aan het einde van de inleiding een overzicht te
% geven van de opbouw van de rest van de tekst. Deze sectie bevat al een aanzet
% die je kan aanvullen/aanpassen in functie van je eigen tekst.

De rest van deze bachelorproef is als volgt opgebouwd:

In Hoofdstuk~\ref{ch:stand-van-zaken} wordt een overzicht gegeven van de stand van zaken binnen het onderzoeksdomein, op basis van een literatuurstudie.

In Hoofdstuk~\ref{ch:methodologie} wordt de methodologie toegelicht en worden de gebruikte onderzoekstechnieken besproken om een antwoord te kunnen formuleren op de onderzoeksvragen.

In Hoofdstuk~\ref{ch:softwareselectie} werden de eisen van IzyCoffee voor een voorraadsysteem geformuleerd. Op basis hiervan werd de meest geschikte software geselecteerd.

In Hoofdstuk~\ref{ch:proof-of-concept} werden ontbrekende functies in de geselecteerde software geïdentificeerd. Aanpassingen aan de code van het geselecteerde programma werden ontwikkeld en geïmplementeerd om de functionaliteit uit te breiden en zo de bedrijfsdoelen beter te bereiken.

In Hoofdstuk~\ref{ch:conclusie}, tenslotte, wordt de conclusie gegeven en een antwoord geformuleerd op de onderzoeksvragen. Daarbij wordt ook een aanzet gegeven voor toekomstig onderzoek binnen dit domein.