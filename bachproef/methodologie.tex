%%=============================================================================
%% Methodologie
%%=============================================================================

\chapter{\IfLanguageName{dutch}{Methodologie}{Methodology}}%
\label{ch:methodologie}

%% TODO: In dit hoofstuk geef je een korte toelichting over hoe je te werk bent
%% gegaan. Verdeel je onderzoek in grote fasen, en licht in elke fase toe wat
%% de doelstelling was, welke deliverables daar uit gekomen zijn, en welke
%% onderzoeksmethoden je daarbij toegepast hebt. Verantwoord waarom je
%% op deze manier te werk gegaan bent.
%% 
%% Voorbeelden van zulke fasen zijn: literatuurstudie, opstellen van een
%% requirements-analyse, opstellen long-list (bij vergelijkende studie),
%% selectie van geschikte tools (bij vergelijkende studie, "short-list"),
%% opzetten testopstelling/PoC, uitvoeren testen en verzamelen
%% van resultaten, analyse van resultaten, ...
%%
%% !!!!! LET OP !!!!!
%%
%% Het is uitdrukkelijk NIET de bedoeling dat je het grootste deel van de corpus
%% van je bachelorproef in dit hoofstuk verwerkt! Dit hoofdstuk is eerder een
%% kort overzicht van je plan van aanpak.
%%
%% Maak voor elke fase (behalve het literatuuronderzoek) een NIEUW HOOFDSTUK aan
%% en geef het een gepaste titel.

Het doel van de vergelijkende analyse is om de meest geschikte bestaande software voor inventarisatie en logistiek van een keten van koffiebars te bepalen.

Hiertoe is het werk verdeeld in de volgende fasen: literatuuronderzoek, bepalen van de meest geschikte software, implementeren van PoC en conclusie 

\section{Softwareselectie voor het voorraadsysteem}

Het werk begon met het definiëren van vereisten voor een inventarisatiesysteem voor de koffieketen IzyCoffee.

Vervolgens werden de vereisten gerangschikt met behulp van de MoSCoW-methode. Deze werd gebruikt om elke vereiste te prioriteren. Dit creëerde een meetbare basis voor de selectie van de meest geschikte software.

De volgende stap was het opstellen van een lange lijst met bestaande voorraadbeheersoftware.

Vervolgens werden de drie programma's die het beste aan de vereisten voldeden, geselecteerd en vergeleken.
Dit bepaalde welke software het meest geschikt was voor de bedrijfsdoelstellingen van de koffieketen IzyCoffee.


\section{Proof of concept}
In deze fase werden de functies van de geselecteerde software geëvalueerd en vergeleken met de eisen van het bedrijf. Vereisten die niet door de functionaliteit van de geselecteerde software werden gedekt, werden geïdentificeerd. Om deze hiaten op te vullen, werden oplossingsvoorstellen ontwikkeld en gepresenteerd.
De laatste fase van deze fase was de ontwikkeling van een proof-of-concept (PoC). Om de geselecteerde software te optimaliseren en te zorgen dat deze voldoet aan beschreven eisen, werden wijzigingen in de code ontwikkeld en geïmplementeerd.

Dit resulteerde in een product dat optimaal voldoet aan de eisen van de speciaalzaak IzyCoffee en rekening houdt met de specifieke kenmerken van deze specifieke onderneming.

\section{Conclusie}
In deze sectie worden de bevindingen van het uitgevoerde werk samengevat. Er wordt een vergelijking gemaakt met de beoogde conclusies in het huidige werk.
Ten slotte worden suggesties voor verdere verbetering en uitbreiding van de software geformuleerd en gepresenteerd.


