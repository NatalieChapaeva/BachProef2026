\chapter{\IfLanguageName{dutch}{Stand van zaken}{State of the art}}%
\label{ch:stand-van-zaken}

% Tip: Begin elk hoofdstuk met een paragraaf inleiding die beschrijft hoe
% dit hoofdstuk past binnen het geheel van de bachelorproef. Geef in het
% bijzonder aan wat de link is met het vorige en volgende hoofdstuk.

% Pas na deze inleidende paragraaf komt de eerste sectiehoofding.


In 2015 stelden de Verenigde Naties Duurzame Ontwikkelingsdoelstelling 12.3 vast, die ernaar streeft de voedselverspilling per hoofd van de bevolking tegen 2030 te halveren ~\autocite{VN2024}. Deze beslissing onderstreept het belang van het aanpakken van voedselverspilling om voedselzekerheid te waarborgen, hulpbronnen duurzaam te gebruiken en de negatieve impact op het milieu te verminderen.

In het kader van deze doelstelling zijn verschillende sectoren, waaronder de horeca, begonnen met het implementeren van maatregelen die gericht zijn op het verminderen van voedselverspilling. Een studie gepubliceerd in het Journal of Cleaner Production \textcite{Cardenas2024} analyseert circulaire praktijken in de horeca, zoals het recyclen van restjes en menu-optimalisatie, die bijdragen aan het verminderen van afval en het bevorderen van duurzame consumptie.

Daarnaast heeft de Wereldtoerismeorganisatie een wereldwijde routekaart ontwikkeld voor het verminderen van voedselverspilling in de toeristische sector, met aanbevelingen voor het implementeren van duurzame praktijken in hotels en restaurants, waaronder personeelstraining en het gebruik van digitale tools om afval te monitoren \parencite{2023}.

Digitale transformatie speelt een sleutelrol in deze inspanningen. Een bibliometrisch onderzoek gepubliceerd in het International Journal of Hospitality Management identificeerde vier belangrijke onderzoeksgebieden op het gebied van digitale transformatie in de horeca: digitale adoptie, impact op stakeholders, determinanten van online aankopen en analyse van online reviews ~\textcite{Peng2024}. Deze gebieden benadrukken het belang van digitale oplossingen, zoals geavanceerde voorraadbeheersystemen, voor het verbeteren van de efficiëntie en duurzaamheid van horecabedrijven.

Het bereiken van de VN-doelstelling om voedselverspilling te verminderen, vereist daarom een ​​alomvattende aanpak die de implementatie van circulaire praktijken en digitale technologieën in de horeca omvat. Dit draagt ​​niet alleen bij aan de bescherming van het milieu, maar verbetert ook de operationele efficiëntie en de duurzaamheid van bedrijven.













%\begin{figure}
%  \centering
%  \includegraphics[width=0.8\textwidth]{grail.jpg}
%  \caption[Voorbeeld figuur.]{\label{fig:grail}Voorbeeld van invoegen van een figuur. Zorg altijd voor een uitgebreid bijschrift dat de figuur volledig beschrijft zonder in de tekst te moeten gaan zoeken. Vergeet ook je bronvermelding niet!}
%\end{figure}

%\begin{listing}
%  \begin{minted}{python}
%    import pandas as pd
%    import seaborn as sns

%    penguins = sns.load_dataset('penguins')
%    sns.relplot(data=penguins, x="flipper_length_mm", y="bill_length_mm", hue="species")
%  \end{minted}
%  \caption[Voorbeeld codefragment]{Voorbeeld van het invoegen van een codefragment.}
%\end{listing}



%\begin{table}
%  \centering
%  \begin{tabular}{lcr}
%    \toprule
%    \textbf{Kolom 1} & \textbf{Kolom 2} & \textbf{Kolom 3} \\
%    $\alpha$         & $\beta$          & $\gamma$         \\
%    \midrule
%    A                & 10.230           & a                \\
%    B                & 45.678           & b                \\
%    C                & 99.987           & c                \\
%    \bottomrule
%  \end{tabular}
%  \caption[Voorbeeld tabel]{\label{tab:example}Voorbeeld van een tabel.}
%\end{table}

