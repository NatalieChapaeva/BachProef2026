%%=============================================================================
%% Samenvatting
%%=============================================================================

% TODO: De "abstract" of samenvatting is een kernachtige (~ 1 blz. voor een
% thesis) synthese van het document.
%
% Een goede abstract biedt een kernachtig antwoord op volgende vragen:
%
% 1. Waarover gaat de bachelorproef?
% 2. Waarom heb je er over geschreven?
% 3. Hoe heb je het onderzoek uitgevoerd?
% 4. Wat waren de resultaten? Wat blijkt uit je onderzoek?
% 5. Wat betekenen je resultaten? Wat is de relevantie voor het werkveld?
%
% Daarom bestaat een abstract uit volgende componenten:
%
% - inleiding + kaderen thema
% - probleemstelling
% - (centrale) onderzoeksvraag
% - onderzoeksdoelstelling
% - methodologie
% - resultaten (beperk tot de belangrijkste, relevant voor de onderzoeksvraag)
% - conclusies, aanbevelingen, beperkingen
%
% LET OP! Een samenvatting is GEEN voorwoord!

%%---------- Nederlandse samenvatting -----------------------------------------
%
% TODO: Als je je bachelorproef in het Engels schrijft, moet je eerst een
% Nederlandse samenvatting invoegen. Haal daarvoor onderstaande code uit
% commentaar.
% Wie zijn bachelorproef in het Nederlands schrijft, kan dit negeren, de inhoud
% wordt niet in het document ingevoegd.

\IfLanguageName{english}{%
\selectlanguage{dutch}
\chapter*{Samenvatting}
\lipsum[1-4]
\selectlanguage{english}
}{}

%%---------- Samenvatting -----------------------------------------------------
% De samenvatting in de hoofdtaal van het document

\chapter*{\IfLanguageName{dutch}{Samenvatting}{Abstract}}

 Dit proefschrift is gewijd aan het vinden van de optimale oplossing voor voorraadbeheer in de horeca (Horeca: hotels, restaurants, cafés). Dit werk is uitgevoerd aan de hand van het voorbeeld van de keten van speciaalzaken in koffiehuizen IzyCoffee. De resultaten van het onderzoek kunnen worden gebruikt door andere bedrijven die een voorraadsysteem moeten verbeteren/implementeren. De relevantie van het project is te danken aan het gebrek aan aangepaste oplossingen voor kleine bedrijven in deze sector. Het grootste probleem is de inefficiëntie van de boekhouding van producten en ingrediënten, wat leidt tot een daling van de winst en een toename van voedselverspilling. Het doel van het project is om voorraadbeheer te vereenvoudigen en te rationaliseren. De methodologie omvat een literatuuronderzoek, het formuleren van eisen voor het voorraadsysteem, een analyse van bestaande oplossingen en de selectie van de meest geschikte optie. Verwacht wordt dat dit resulteert in het bepalen van de software die het beste voldoet aan de geformuleerde eisen en praktische voorstellen voor het aanpassen en verbeteren van het voorraadsysteem. De resultaten van het onderzoek worden gepresenteerd als een vergelijkende demonstratie van de voorgestelde oplossing en de momenteel gebruikte methoden.